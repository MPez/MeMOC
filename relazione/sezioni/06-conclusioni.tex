%\clearpage
\section{Conclusioni}

\begin{figure}[htb]
	\centering
	\includegraphics[width=.85\textwidth]{immagini/cplex_tabu_tempo.pdf}
	\caption{Tempi istanze casuali - \acronimo{cplex} vs. \tabu}
	\label{fig:tempi cplex tabu}
\end{figure}

In figura \ref{fig:tempi cplex tabu} si può vedere come i tempi di calcolo dell'euristica \tabu siano migliori di almeno due unità di grandezza rispetto a quelli ottenuti con \acronimo{cplex}: questo giustifica la perdita di precisione sui valori delle soluzioni trovati, soprattutto in contesti dove sia critico il tempo di calcolo e non sia così importante la soluzione esatta del problema.

Per confrontare l'efficacia dell'implementazione della \tabu rispetto a \acronimo{cplex} ho calcolato prima la media sulle ripetizioni delle istanze identiche e poi la media sulle istanze simili con lo stesso numero di nodi ottenendo i grafici in figura \ref{fig:costi cplex tabu}: si nota che il valore della soluzione ottenuta con la metaeuristica non si allontana troppo dal valore esatto, con un basso numero di nodi, il peggioramento, si attesta intorno al $2\%$ e cresce al crescere della loro quantità.
Aumentando il valore di \emph{tabu tenure} si riesce però a mitigare la perdita di precisione recuperando circa l'$1\%$ e questo conferma la necessità di effettuare un'attenta calibrazione dei parametri con lo scopo di ottenere risultati migliori.
In figura sono riportate anche le medie dei valori peggiori delle soluzioni che sono stati calcolati con l'algoritmo e indicano come in alcuni casi la \tabu non riesca ad avvicinarsi sufficientemente all'ottimo globale incappando probabilmente in qualche ottimo locale.

\begin{figure}[htb]
	\centering
	\begin{subfigure}[b]{.45\textwidth}
			\includegraphics[width=\textwidth]{immagini/cplex_tabu4_compare.pdf}
			\caption{Tabu tenure 4}
	\end{subfigure}
	\quad
	\begin{subfigure}[b]{.45\textwidth}
			\includegraphics[width=\textwidth]{immagini/cplex_tabu5_compare.pdf}
			\caption{Tabu tenure 5}
	\end{subfigure}
	\\
	\begin{subfigure}[b]{.45\textwidth}
			\includegraphics[width=\textwidth]{immagini/cplex_tabu6_compare.pdf}
			\caption{Tabu tenure 6}
	\end{subfigure}
	\quad
	\begin{subfigure}[b]{.45\textwidth}
			\includegraphics[width=\textwidth]{immagini/cplex_tabu7_compare.pdf}
			\caption{Tabu tenure 7}
	\end{subfigure}
	\\
	\begin{subfigure}[b]{.45\textwidth}
			\includegraphics[width=\textwidth]{immagini/cplex_tabu8_compare.pdf}
			\caption{Tabu tenure 8}
	\end{subfigure}
	\caption{Costi medi e massimi istanze casuali - \acronimo{cplex} vs \tabu}
	\label{fig:costi cplex tabu}
\end{figure}

In conclusione posso dire che le mie implementazioni degli algoritmi si comportano come atteso, sia in termini prestazionali che in termini di precisione della soluzione.

La mia versione della \tabu risulta essere leggermente più lenta rispetto alla versione fornita durante i laboratori: questo fatto è dovuto al modo in cui calcolo il vicinato, per permettere la ricerca dei $k$ migliori vicini utilizzo un insieme di mosse ordinato per costo della soluzione, mantenere questa struttura ordinata ad ogni inserimento impatta sui tempi di risoluzione e provoca questo differenza di prestazioni.
Rimane comunque un peggioramento marginale e facilmente eliminabile nel caso si renda necessario.

Il lavoro svolto per sviluppare gli algoritmi mi è stato molto utile per comprendere meglio alcuni aspetti della materia che il solo studio teorico non può sicuramente esaurire. Infatti il poter "toccare" con mano i problemi che si pongono durante l'implementazione di un metodo ed eseguire delle prove sperimentali per testarne le prestazioni, rende molto più appaganti e utili lo studio e la comprensione degli argomenti.