%\clearpage
\section{Conclusioni}

In figura \ref{fig:tempi cplex tabu} si può vedere come i tempi di calcolo dell'euristica \tabu per istanze casuali siano migliori di almeno due unità di grandezza rispetto a quelli ottenuti con \acronimo{cplex}: questo giustifica la perdita di precisione sui valori delle soluzioni trovati, soprattutto in contesti dove sia critico il tempo di calcolo e non sia così importante la soluzione esatta del problema.

\begin{figure}[H]
	\centering
	\includegraphics[width=.7\textwidth]{cplex_tabu_tempo}
	\caption{Tempi istanze casuali - \acronimo{cplex} vs. \tabu}
	\label{fig:tempi cplex tabu}
\end{figure}

Anche per quanto riguarda le istanze cluster, i tempi di calcolo sono a vantaggio di \tabu: si può vedere in figura \ref{fig:tempi cplex tabu cluster} che la differenza tra i 2 algoritmi è minore per le istanze a 1 cluster e con pochi nodi, ma cresce con l'aumentare dei cluster e dei nodi,  rimanendo comunque entro 2 ordini di grandezza.

\begin{figure}[H]
	\centering
	\begin{subfigure}[b]{.45\textwidth}
		\includegraphics[width=\textwidth]{cplex_tabu_cluster_1_tempo}
		\caption{Istanze 1 cluster}
	\end{subfigure}
	\quad
	\begin{subfigure}[b]{.45\textwidth}
		\includegraphics[width=\textwidth]{cplex_tabu_cluster_2_tempo}
		\caption{Istanze 2 cluster}
	\end{subfigure}
	\\
	\begin{subfigure}[b]{.45\textwidth}
		\includegraphics[width=\textwidth]{cplex_tabu_cluster_3_tempo}
		\caption{Istanze 3 cluster}
	\end{subfigure}
	\quad
	\begin{subfigure}[b]{.45\textwidth}
		\includegraphics[width=\textwidth]{cplex_tabu_cluster_4_tempo}
		\caption{Istanze 4 cluster}
	\end{subfigure}
	\caption{Tempi istanze cluster - \acronimo{cplex} vs. \tabu}
	\label{fig:tempi cplex tabu cluster}
\end{figure}

Per confrontare l'efficacia dell'implementazione della \tabu rispetto a \acronimo{cplex} ho calcolato prima la media sulle ripetizioni delle istanze identiche e poi la media sulle istanze simili con lo stesso numero di nodi ottenendo i grafici in figura \ref{fig:costi cplex tabu}: si nota che il valore della soluzione ottenuta con la metaeuristica non si allontana troppo dal valore esatto, con un basso numero di nodi, il peggioramento, si attesta intorno al $2\%$ e cresce al crescere della loro quantità.
Aumentando il valore di \emph{tabu tenure} si riesce però a mitigare la perdita di precisione recuperando circa l'$1\%$ e questo conferma la necessità di effettuare un'attenta calibrazione dei parametri con lo scopo di ottenere risultati migliori.
In figura sono riportate anche le medie dei valori peggiori delle soluzioni che sono stati calcolati con l'algoritmo e indicano come in alcuni casi la \tabu non riesca ad avvicinarsi sufficientemente all'ottimo globale incappando probabilmente in qualche ottimo locale.

\begin{figure}[H]
	\centering
	\begin{subfigure}[b]{.45\textwidth}
			\includegraphics[width=\textwidth]{cplex_tabu4_compare}
			\caption{Tabu tenure 4}
	\end{subfigure}
	\quad
	\begin{subfigure}[b]{.45\textwidth}
			\includegraphics[width=\textwidth]{cplex_tabu8_compare}
			\caption{Tabu tenure 8}
	\end{subfigure}
	\caption{Costi medi e massimi istanze casuali - \acronimo{cplex} vs \tabu}
	\label{fig:costi cplex tabu}
\end{figure}

Per quanto riguarda le istanze cluster si può osservare un comportamento simile a quello delle istanze casuali, infatti al crescere del numero di nodi e della \emph{tabu tenure} si osservano miglioramenti nei costi delle soluzioni trovate.

In particolare, per le istanze a 1 cluster (figura \ref{fig:costi cplex tabu cluster 1}), si nota che la \tabu riesce in media a trovare la soluzione ottima ma esiste una forte variabilità tra istanze simili trovando, a volte, soluzioni molto lontane da quella ottima.

\begin{figure}[H]
	\centering
	\begin{subfigure}[b]{.45\textwidth}
			\includegraphics[width=\textwidth]{cplex_tabu4_cluster1_compare}
			\caption{Tabu tenure 4}
	\end{subfigure}
	\quad
	\begin{subfigure}[b]{.45\textwidth}
			\includegraphics[width=\textwidth]{cplex_tabu8_cluster1_compare}
			\caption{Tabu tenure 8}
	\end{subfigure}
	\caption{Costi medi e massimi istanze 1 cluster - \acronimo{cplex} vs \tabu}
	\label{fig:costi cplex tabu cluster 1}
\end{figure}

Le istanze a 2 cluster (figura \ref{fig:costi cplex tabu cluster 2}) presentano anch'esse forte variabilità nel valore delle soluzioni trovate ma in media si avvicinano molto alla soluzione ottima, migliorando con l'aumentare della \emph{tabu tenure}.

\begin{figure}[H]
	\centering
	\begin{subfigure}[b]{.45\textwidth}
			\includegraphics[width=\textwidth]{cplex_tabu4_cluster2_compare}
			\caption{Tabu tenure 4}
	\end{subfigure}
	\quad
	\begin{subfigure}[b]{.45\textwidth}
			\includegraphics[width=\textwidth]{cplex_tabu8_cluster2_compare}
			\caption{Tabu tenure 8}
	\end{subfigure}
	\caption{Costi medi e massimi istanze 2 cluster - \acronimo{cplex} vs \tabu}
	\label{fig:costi cplex tabu cluster 2}
\end{figure}

Con l'aumentare dei cluster e di conseguenza dei nodi (figure \ref{fig:costi cplex tabu cluster 3} e \ref{fig:costi cplex tabu cluster 4}), si nota una costante perdita di precisione nella soluzione trovata rispetto a quella ottima, che viene mitigata con l'adozione di valori di \emph{tabu tenure} più alta.

\begin{figure}[H]
	\centering
	\begin{subfigure}[b]{.45\textwidth}
			\includegraphics[width=\textwidth]{cplex_tabu4_cluster3_compare}
			\caption{Tabu tenure 4}
	\end{subfigure}
	\quad
	\begin{subfigure}[b]{.45\textwidth}
			\includegraphics[width=\textwidth]{cplex_tabu8_cluster3_compare}
			\caption{Tabu tenure 8}
	\end{subfigure}
	\caption{Costi medi e massimi istanze 3 cluster - \acronimo{cplex} vs \tabu}
	\label{fig:costi cplex tabu cluster 3}
\end{figure}

\begin{figure}[H]
	\centering
	\begin{subfigure}[b]{.45\textwidth}
			\includegraphics[width=\textwidth]{cplex_tabu4_cluster4_compare}
			\caption{Tabu tenure 4}
	\end{subfigure}
	\quad
	\begin{subfigure}[b]{.45\textwidth}
			\includegraphics[width=\textwidth]{cplex_tabu8_cluster4_compare}
			\caption{Tabu tenure 8}
	\end{subfigure}
	\caption{Costi medi e massimi istanze 4 cluster - \acronimo{cplex} vs \tabu}
	\label{fig:costi cplex tabu cluster 4}
\end{figure}

Per una visione più esaustiva dei dati raccolti si rimanda all'appendice dove vengono riportati tutti i grafici (figure da \ref{fig:costi cplex tabu cluster 1 completo} a \ref{fig:costi cplex tabu cluster 4 completo}) e tutte le tabelle (da \ref{tab:cplex tabu cluster 1} a \ref{tab:cplex tabu cluster 4}) relativi alla comparazione tra \acronimo{cplex} e \tabu per le istanze cluster.

In conclusione posso dire che le mie implementazioni degli algoritmi si comportano come atteso, sia in termini prestazionali che in termini di precisione della soluzione.

La mia versione della \tabu risulta essere leggermente più lenta rispetto alla versione fornita durante i laboratori: questo fatto è dovuto al modo in cui calcolo il vicinato, per permettere la ricerca dei $k$ migliori vicini utilizzo un insieme di mosse ordinato per costo della soluzione, mantenere questa struttura ordinata ad ogni inserimento impatta sui tempi di risoluzione e provoca questo differenza di prestazioni.
Rimane comunque un peggioramento marginale e facilmente eliminabile nel caso si renda necessario.

Il lavoro svolto per sviluppare gli algoritmi mi è stato molto utile per comprendere meglio alcuni aspetti della materia che il solo studio teorico non può sicuramente esaurire. Infatti il poter "toccare" con mano i problemi che si pongono durante l'implementazione di un metodo ed eseguire delle prove sperimentali per testarne le prestazioni, rende molto più appaganti e utili lo studio e la comprensione degli argomenti.