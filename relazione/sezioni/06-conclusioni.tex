%\clearpage
\section{Conclusioni}

\begin{figure}[htb]
	\centering
	\includegraphics[width=.85\textwidth]{immagini/cplex_tabu_tempo.pdf}
	\caption{Tempi istanze casuali - \acronimo{cplex} vs. \tabu}
	\label{fig:tempi cplex tabu}
\end{figure}

In figura \ref{fig:tempi cplex tabu} si può vedere come i tempi di calcolo dell'euristica \tabu siano migliori di almeno due unità di grandezza rispetto a quelli trovati con \acronimo{cplex}: questo giustifica la perdita di precisione sui valori delle soluzioni trovati, come si evince in figura \ref{fig:costi cplex tabu}, soprattutto in contesti dove sia critico il tempo di calcolo e non sia così importante la soluzione esatta del problema.

\begin{figure}[htb]
	\centering
	\includegraphics[width=.85\textwidth]{immagini/cplex_tabu_costo.pdf}
	\caption{Costi istanze casuali - \acronimo{cplex} vs. \tabu}
	\label{fig:costi cplex tabu}
\end{figure}

In conclusione posso dire che le mie implementazioni degli algoritmi si comportano come atteso, sia in termini prestazionali che in termini di precisione della soluzione.

La mia versione della \tabu risulta essere leggermente più lenta rispetto alla versione fornita durante i laboratori: questo fatto è dovuto al modo in cui calcolo il vicinato, per permettere la ricerca dei $k$ migliori vicini utilizzo un insieme di mosse ordinato per costo della soluzione, mantenere questa struttura ordinata ad ogni inserimento impatta sui tempi di risoluzione e provoca questo differenza di prestazioni.
Resta comunque un peggioramento marginale e facilmente eliminabile nel caso sia necessario.