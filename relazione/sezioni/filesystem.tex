%disegna la struttura di cartelle presente nell'archivio consegnato
\tikzstyle{every node}=[thick,anchor=west, rounded corners, font={\scriptsize\ttfamily}, inner sep=2.5pt]
\tikzstyle{selected}=[draw=blue,fill=blue!10]
\tikzstyle{root}=[selected, fill=blue!30]
\tikzstyle{optional}=[dashed, draw=blue, fill=blue!5]

\begin{tikzpicture}[%
    scale=.7,
    grow via three points={one child at (0.5,-0.65) and
    two children at (0.5,-0.65) and (0.5,-1.1)},
    edge from parent path={(\tikzparentnode.south) |- (\tikzchildnode.west)}]
  \node [root] {eserc.lab.01}
    child { node [selected] {eserc.lab.02}
      child { node {istanza.cpp}}
      child { node {istanza.cpp}}
      child { node {main.cpp}}
      child { node {Makefile}}
      child { node {mossa.cpp}}
      child { node {mossa.h}}
      child { node {solutore.cpp}}
      child { node {solutore.h}}
      child { node {soluzione.cpp}}
      child { node {soluzione.h}}
    }
    child { node at (0,-5) [optional] {instances}
      child { node {\dots}}
    }
    child { node at (0,-6) [optional] {results}
      child { node {\dots}}
    }
    child { node at (0,-7) [optional] {solutions}
      child { node {\dots}}
    }
    child { node at (0,-8) {cpxmacro.h}}
    child { node at (0,-8.2) {eserc\_lab.sh}}
    child { node at (0,-8.4) {generator.py}}
    child { node at (0,-8.6) {main.cpp}}
    child { node at (0,-8.8) {Makefile}}
    child { node at (0,-9) {pannello.cpp}}
    child { node at (0,-9.2) {pannello.h}}
    child { node at (0,-9.4) {plot.gnuplot}}
    child { node at (0,-9.6) {statistic.py}};
\end{tikzpicture}