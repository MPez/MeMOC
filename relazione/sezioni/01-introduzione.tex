\section{Introduzione}
Con la presente relazione intendo descrivere in modo dettagliato il lavoro svolto per completare le esercitazioni dell'insegnamento \memoclong.

Il problema proposto consiste nell'individuare la sequenza di foratura ottimale che minimizzi i tempi di produzione su dei pannelli forati prodotti da un'azienda metalmeccanica.
Tale problema è rappresentabile mediante un grafo pesato completo in cui l'insieme di nodi e archi identificano rispettivamente i fori e il tragitto da percorrere per raggiungere la posizione di ogni foro: si tratta quindi di trovare il cammino di costo minore tale da visitare ogni nodo del grafo una volta sola.

Si deve quindi risolvere un problema del commesso viaggiatore (\acronimo{tsp}), modellato su di una rete di flusso in cui ogni nodo riceve una sola unità di flusso, viene visitato una e una sola volta e il costo del cammino sia minimo.
Il modello di programmazione matematica è stato fornito come specifica ed è stato richiesto di implementare tale modello tramite l'utilizzo delle \acronimo{api} di \acronimo{cplex} e tramite un metodo a scelta dello studente tra quelli presentati a lezione.

La mia scelta è caduta sulla metaeuristica \tabu: nel seguito descriverò come ho progettato e sviluppato tale algoritmo descrivendo il codice prodotto e gli strumenti che ho utilizzato.
Dopo aver realizzato i due solutori ho effettuato dei test su alcune istanze di prova, generate in modo automatico e casuale, in modo da poter valutare la bontà delle implementazioni e la loro efficacia: i metodi usati sono stati esaminati singolarmente e poi confrontati per ottenere una comparativa sui risultati ottenuti.

I risultati vengono riportati mediante grafici e tabelle in modo da consentire un raffronto visuale ma anche un'eventuale valutazione puntuale.