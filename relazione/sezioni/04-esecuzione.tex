%\clearpage
\section{Esecuzione}
L'esecuzione del programma è gestita mediante lo script \emph{bash} chiamato \script{eserc\_lab.sh}; tale script è stato dotato di diverse opzioni che permettono di eseguire il programma con diverse modalità richiamabili tramite gli appositi flag:

\begin{description}
	\item[\texttt{-a arg}]: modalità \modo{all}, esegue in sequenza tutte le operazioni descritte nella sezione \ref{sec:bash} e comprende tutte le altre modalità di esecuzione; permette quindi di eseguire l'intero programma ed ottenere tutti i risultati; l'argomento \texttt{arg} è necessario per impostare il numero di istanze da generare in quanto indica il numero di ripetizioni da effettuare;
	\item[\texttt{-c arg}]: modalità \modo{casuali}, richiama lo script \script{generator.py} per generare solamente i problemi creati casualmente; l'argomento \texttt{arg} serve per impostare il numero di problemi da generare;
	\item[\texttt{-g}]: modalità \modo{gnuplot}, richiama \emph{gnuplot} per disegnare i grafici con i dati ottenuti;
	\item[\texttt{-i arg}]: modalità \modo{circolari}, richiama lo script \script{generator.py} per generare  solamente i problemi circolari; l'argomento \texttt{arg} serve per impostare il numero di problemi da generare;
	\item[\texttt{-l arg}]: modalità \modo{cluster}, richiama lo script \script{generator.py} per generare  solamente i problemi a cluster; l'argomento \texttt{arg} serve per impostare il numero di problemi da generare;
	\item[\texttt{-r arg}]: modalità \modo{cplex}, richiama il programma \script{cplex} per risolvere i problemi che sono stati generati; l'argomento \texttt{arg} serve per indicare il numero di problemi da risolvere;
	\item[\texttt{-s}]: modalità \modo{statistiche}, richiama lo script \script{statistic.py} per elaborare i risultati ottenuti e generare i file di input per \emph{gnuplot};
	\item[\texttt{-t arg}]: modalità \modo{tabu search}, richiama il programma \script{tabusearch} per risolvere i problemi generati; l'argomento \texttt{arg} serve per indicare il numero di problemi da risolvere.
\end{description}

Per valutare e confrontare l'efficienza dei due metodi in modo significativo è necessario avere una considerevole quantità di dati, questo implica il dover generare e risolvere molte istanze diverse.

Per questo motivo ho deciso di creare tre diversi tipi di problemi: istanze con distribuzione di nodi casuale all'interno della griglia, con una disposizione a cluster, suddividendo la griglia in quarti e concentrando la disposizione dei nodi prima su un quarto, poi su due, tre e quattro cluster in modo da capire se ci sono differenze a parità di densità di nodi: ho utilizzato infatti come indice per creare i nodi la densità in funzione delle dimensioni della griglia, o del cluster.

Come ultima tipologia di istanza ho usato quella circolare che, su una griglia con nodi equispaziati, equivale ad un quadrato con i vertici in direzione degli assi dei lati della griglia stessa.

Visto che la distribuzione dei nodi all'interno della griglia avviene in modo casuale sui problemi \emph{casuali} e su quelli \emph{cluster}, ho deciso di creare più problemi simili, cioè con lo stesso numero di nodi, per tipo; il numero di problemi è impostabile tramite l'argomento \texttt{arg} delle opzioni dello script; inoltre per creare istanze con un numero di nodi crescente utilizzo un paramentro \texttt{dens}, che rappresenta la densità dei nodi, che viene incrementato durante la generazione delle istanze.

\subsection{Struttura cartelle}
Affinché l'esecuzione dei programmi non provochi errori è necessario che la struttura di file sia coerente con quella usata nell'ambiente di sviluppo; insieme alla presente relazione viene fornito un archivio contenente tutti i file sviluppati necessari al funzionamento, tralasciando, ovviamente, quelli che verranno creati dagli stessi programmi.
In figura \ref{fig:struttura} viene descritta la struttura di file e cartelle nell'ambiente di sviluppo.

\begin{figure}[htb]
\centering
%disegna la struttura di cartelle presente nell'archivio consegnato
\tikzstyle{every node}=[thick,anchor=west, rounded corners, font={\scriptsize\ttfamily}, inner sep=2.5pt]
\tikzstyle{selected}=[draw=blue,fill=blue!10]
\tikzstyle{root}=[selected, fill=blue!30]
\tikzstyle{optional}=[dashed, draw=blue, fill=blue!5]

\begin{tikzpicture}[%
    scale=.7,
    grow via three points={one child at (0.5,-0.65) and
    two children at (0.5,-0.65) and (0.5,-1.1)},
    edge from parent path={(\tikzparentnode.south) |- (\tikzchildnode.west)}]
  \node [root] {eserc.lab.01}
    child { node [selected] {eserc.lab.02}
      child { node {istanza.cpp}}
      child { node {istanza.cpp}}
      child { node {main.cpp}}
      child { node {Makefile}}
      child { node {mossa.cpp}}
      child { node {mossa.h}}
      child { node {solutore.cpp}}
      child { node {solutore.h}}
      child { node {soluzione.cpp}}
      child { node {soluzione.h}}
    }
    child { node at (0,-5) [optional] {instances}
      child { node {\dots}}
    }
    child { node at (0,-6) [optional] {results}
      child { node {\dots}}
    }
    child { node at (0,-7) [optional] {solutions}
      child { node {\dots}}
    }
    child { node at (0,-8) {cpxmacro.h}}
    child { node at (0,-8.2) {eserc\_lab.sh}}
    child { node at (0,-8.4) {generator.py}}
    child { node at (0,-8.6) {main.cpp}}
    child { node at (0,-8.8) {Makefile}}
    child { node at (0,-9) {pannello.cpp}}
    child { node at (0,-9.2) {pannello.h}}
    child { node at (0,-9.4) {plot.gnuplot}}
    child { node at (0,-9.6) {statistic.py}};
\end{tikzpicture}
\caption{Struttura file e cartelle delle esercitazioni}
\label{fig:struttura}
\end{figure}

Le cartelle tratteggiare sono quelle che conterranno i file creati dai programmi e non saranno quindi presenti inizialmente; nella cartella \emph{instances} saranno salvati due file per ogni istanza generata, uno contenente la matrice dei costi e l'altro la griglia con la disposizione dei nodi.

Nella cartella \emph{results} verranno salvati i file relativi ai risultati ottenuti dalle risoluzioni dei problemi e i \acronimo{pdf} con i grafici; nella cartella \emph{solutions} ci saranno le soluzioni ai problemi generate dal programma \script{cplex}.

\subsection{Compilazione}
Una volta aperto l'archivio contenente il codice sorgente ed essersi assicurati che siano installati e presenti nella variabile di sistema \comando{path} le librerie e i software descritti nella sezione \ref{sec:ambiente} e, prima di eseguire il programma tramite script \emph{bash}, è necessario procedere alla compilazione del codice \acronimo{c$++$}: per fare questo è sufficiente digitare ed eseguire il comando \script{make} all'interno della cartella radice del progetto, per compilare \script{cplex}, ed eseguire nuovamente il comando \script{make} all'interno della cartella \script{eserc.lab.02} per compilare \script{tabusearch}.

