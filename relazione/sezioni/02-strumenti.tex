%\clearpage
\section{Strumenti utilizzati}
Per la realizzazione delle esercitazioni mi sono avvalso di diversi strumenti e linguaggi di programmazione: alcuni usati per l'effettiva implementazione dei metodi e altri per automatizzare le procedure di creazione delle istanze, della loro risoluzione e della realizzazione dei grafici.
Di seguito verranno descritti gli strumenti che ho utilizzato e le motivazioni che mi hanno guidato nella loro scelta.

\begin{description}
	\item[\textsc{C$++$}]: linguaggio di programmazione orientato agli oggetti che ho utilizzato per l'implementazione del modello per la risoluzione mediante il metodo del simplesso e per lo sviluppo della metaeuristica \tabu;
	\item[\textsc{Make}]: per automatizzare il processo di compilazione dei sorgenti e generazione dei file eseguibili sono stati utilizzati due \emph{makefile} diversi, uno per esercitazione;
	\item[\textsc{CPLEX}]: software di ottimizzazione implementato nel linguaggio \acronimo{C} in grado di risolvere problemi di programmazione lineare e lineare intera utilizzando le varianti primale e duale del metodo del simplesso. Fornisce delle \acronimo{api} verso diversi linguaggi, tra cui il \acronimo{c$++$};
	\item[\textsc{Python}]: linguaggio multiparadigma, semplice e flessibile che consente di sviluppare script in poco tempo e, grazie alla notevole quantità librerie esistenti, consente di scrivere anche applicazioni complesse.
	
	Ho utilizzato questo linguaggio per la creazione delle istanze del problema in quanto non era pensabile produrre a mano un numero ragionevole di problemi; inoltre è stato utilizzato per analizzare i risultati ottenuti e produrre i dati statistici da visualizzare sui grafici;
	\item[\textsc{Gnuplot}]: programma che consente di disegnare grafici mediante l'utilizzo di funzioni matematiche o di dati grezzi sia in \acronimo{2d} che in \acronimo{3d}; è semplice da utilizzare ma è dotato di molte opzioni per personalizzare la resa dei grafici e la visualizzazione dei dati;
	\item[\textsc{Bash}]: per rendere il processo di creazione delle istanze, risoluzione mediante i due algoritmi e analisi dei dati automatico, ho realizzato uno script bash che tramite diverse opzioni consente di eseguire le operazioni desiderate.
\end{description}

\subsection{Ambiente di sviluppo}
\label{sec:ambiente}
I programmi sono state sviluppati sul \acronimo{pc} portatile in mio possesso dotato delle seguenti caratteristiche:
\begin{description}
\item[\textsc{processore}]: Intel Core i7-2670QM @ 2.20GHz
\item[\textsc{ram}]: 6\acronimo{gb}
\end{description}

Di seguito sono riportate le versioni di software, linguaggi e librerie utilizzate:
\begin{description}
\item[\textbf{C$++$}]: \acronimo{c$++$03}
\item[\textbf{Make}]: 4.1
\item[\textbf{CPLEX}]: 12.6.0
\item[\textbf{Python}]: 3.4.2
\item[\textbf{Gnuplot}]: 4.6
\item[\textbf{Bash}]: 4.3.33
\end{description}

I programmi sviluppati sono stati testati anche presso la macchina del laboratorio disponibile mediante \emph{ssh} all'indirizzo \script{torre.studenti.math.unipd.it} e sono quindi conformi alle specifiche date.